\subsection{\texttt{configuration.ccl}}

\todo{Talk about requirements for bringing carpetx things in for accessing the APIs}

\subsection{\texttt{interface.ccl}}

\subsubsection{Tags}

Interface files declare a thorn's \Cactus-facing interface by defines grid functions, variables and others (see Sec.~C1.2 of the \Cactus user guide for more details). Drivers, such as \CarpetX, are able to parse the \texttt{TAGS} string in \texttt{interface.ccl} declarations in order to have its behavior customized.

Tags are declared in interface statements with the \texttt{TAGS=<tags>} syntax. The \texttt{<tags>} declaration consists of a single quote string (marked by \texttt{'}) with space separated key-value pars of the form \texttt{key="value"}. For example, a tagged interface declaration will be similar to the following
%
\begin{lstlisting}[language=bash]
  CCTK_<type> <group-name> TAGS='key_1="key1" key_2="key2" ...'
  {
    ...
  } "A group of real variables"
\end{lstlisting}

In this section, we will list the tags supported by \CarpetX\space and describe their usage.

\begin{enumerate}
  \item \texttt{rhs}:
  \item \texttt{dependents}:
  \item \texttt{checkpoint}:
\end{enumerate}

\subsection{\texttt{schedule.ccl}}

\todo{Talk about important new schedule bins and the read/write safeguards on schedules.}

\subsection{Preprocessor macros}

\todo{Talk about new macros and new macro features. The text below is mostly ok, but needs to be adapted}

The macros \texttt{DECLARE\_CCTK\_ARGUMENTSX\_WaveToyX\_RHS}, \texttt{CCTK\_ARGUMENTS} and \texttt{DECLARE\_CCTK\_PARAMETERS} allow the thorn writer to access parameters and grid functions declared in the thorn's \texttt{.ccl} files. Note that \Cactus\space now supports the \texttt{DECLARE\_CCTK\_ARGUMENTSX\_FUNC\_NAME} macro, where \texttt{FUNC\_NAME} is the name of a grid function declared in the \texttt{schedule.ccl} file. These macros restrict the access of a function to it's schedule-declared grid functions. More importantly, it provides a variable called \texttt{grid} which can be used to access the functionalities of the \texttt{Loop} API.