The recommended way of performing interpolation is using the flesh interface for interpolation by calling \texttt{CCTK\_InterpGridArrays()}, as documented on Sec.~C1.7.4 of the \Cactus\space user guide and with detailed API description in Sec.~A149 of the \Cactus\space reference manual.

Internally, this functionality is implemented by the \texttt{DriverInterpolate} function. If users desire, they can call it directly instead of using \texttt{CCTK\_InterpGridArrays()}. In order to do so, users must add the following to their \texttt{interface.ccl} file
%
\begin{lstlisting}[language=bash]
  CCTK_INT FUNCTION DriverInterpolate(
    CCTK_POINTER_TO_CONST IN cctkGH,
    CCTK_INT IN N_dims,
    CCTK_INT IN local_interp_handle,
    CCTK_INT IN param_table_handle,
    CCTK_INT IN coord_system_handle,
    CCTK_INT IN N_interp_points,
    CCTK_INT IN interp_coords_type,
    CCTK_POINTER_TO_CONST ARRAY IN interp_coords,
    CCTK_INT IN N_input_arrays,
    CCTK_INT ARRAY IN input_array_indices,
    CCTK_INT IN N_output_arrays,
    CCTK_INT ARRAY IN output_array_types,
    CCTK_POINTER ARRAY IN output_arrays)
  REQUIRES FUNCTION DriverInterpolate
\end{lstlisting}

Not that \texttt{DriverInterpolate} and \texttt{CCTK\_InterpGridArrays()} receive exactly the same parameters. Again, a complete description of these is given in Sec.~A149 of the \Cactus\space reference manual. Additionally, \CarpetX\space provides a method called \texttt{Interpolate}, with the following signature
%
\begin{lstlisting}
  void Interpolate(
    const CCTK_POINTER_TO_CONST cctkGH_,
    const CCTK_INT npoints,
    const CCTK_REAL *restrict const globalsx,
    const CCTK_REAL *restrict const globalsy,
    const CCTK_REAL *restrict const globalsz,
    const CCTK_INT nvars,
    const CCTK_INT *restrict const varinds,
    const CCTK_INT *restrict const operations
    const CCTK_INT allow_boundaries,
    const CCTK_POINTER resultptrs_
  )
\end{lstlisting}
%
The parameters are \todo{(parameters need to be actually filled)}
%
\begin{itemize}
  \item \texttt{const CCTK\_POINTER\_TO\_CONST cctkGH\_}: Pointer to the \Cactus\space \textit{grid hierarchy}. This is implicitly provided to all functions using the \texttt{DECLARE\_CCTK\_ARGUMENTS\_*} family of macros via the \texttt{cctkGH} variable.

  \item \texttt{const CCTK\_INT npoints}:
  \item \texttt{const CCTK\_REAL *restrict const globalsx}:
  \item \texttt{const CCTK\_REAL *restrict const globalsy}:
  \item \texttt{const CCTK\_REAL *restrict const globalsz}:
  \item \texttt{const CCTK\_INT nvars}:
  \item \texttt{const CCTK\_INT *restrict const varinds}:
  \item \texttt{const CCTK\_INT *restrict const operations}:
  \item \texttt{const CCTK\_INT allow\_boundaries}:
  \item \texttt{const CCTK\_POINTER resultptrs\_}:
\end{itemize}

In order to call it, users should add the following to their \texttt{interface.ccl} files
%
\begin{lstlisting}
  void FUNCTION Interpolate(
    CCTK_POINTER_TO_CONST IN cctkGH,
    CCTK_INT IN npoints,
    CCTK_REAL ARRAY IN coordsx,
    CCTK_REAL ARRAY IN coordsy,
    CCTK_REAL ARRAY IN coordsz,
    CCTK_INT IN nvars,
    CCTK_INT ARRAY IN varinds,
    CCTK_INT ARRAY IN operations,
    CCTK_INT IN allow_boundaries,
    CCTK_POINTER IN resultptrs)
  REQUIRES FUNCTION Interpolate
\end{lstlisting} 

Lastly, to control the order of the interpolation operators, users can set the \texttt{CarpetX::interpolation\_order} parameter, which defaults to 1 and must be greater than 0.