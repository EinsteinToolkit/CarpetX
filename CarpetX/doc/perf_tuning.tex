\subsection{Parallelism model}

When running on CPUs, \AMReX\space will divide the grid in boxes whose sizes are controlled by \texttt{blocking\_factor} and \texttt{max\_grid\_size} parameter families. These will be distributed over MPI ranks and each rank will subdivide the boxes on \textit{tiles}. This subdivision into tiles is only logical (not in memory). Threads may then process tiles in any random order. When running on GPU's, tiling is disabled and \AMReX\space tries to assign one GPU per MPI process. \todo{If there are more MPI processes than grids to distribute, some MPI process may become idle. To prevent this from happening, the FLAG flag can be passed to \AMReX\space via PARAMETER to ensure that each MPI rank processes at least one grid.}

\subsection{Parallelism control parameters}

\CarpetX\space allows users to control \texttt{AMReX}'s underlying load balancing mechanism by exposing \texttt{AMReX} parameters to \texttt{Cactus} parameter files. We shall now describe these parameters and their functionality.

\subsubsection{\texttt{CarpetX::max\_grid\_size\_\{xyz\}}}

\AMReX\space will divide the domain in every direction so that each grid is no longer than \texttt{max\_grid\_size} in that direction. It defaults to 32 in each coordinate direction.

\subsubsection{\texttt{CarpetX::blocking\_factor\_\{xyz\}}}

Constrain grid creation so that each grid must be divisible by \texttt{blocking\_factor}. This implies that both the domain (at each level) and \texttt{max\_grid\_size} must be divisible by \texttt{blocking\_factor}, and that \texttt{blocking\_factor} must be either 1 or a power of 2. It defaults to 8 in every direction.

\subsubsection{\texttt{CarpetX::refine\_grid\_layout}}

If set to \texttt{"yes"} and the number of grids created is less than the number of processors ($N_\text{grids} < N_\text{procs}$), then grids will be further subdivided until $N_\text{grids} \geq N_\text{procs}$. If subdividing the grids to achieve $N_\text{grids} \geq N_\text{procs}$ would violate the \texttt{blocking\_factor} criterion then additional grids are not created and the number of grids will remain less than the number of processors. It defaults to \texttt{"yes"}

\subsubsection{\texttt{CarpetX::grid\_efficiency}}

Because mesh refinement in \AMReX is done per-cell (see sec.~\ref{sec:advanced_amr} for further details), a refinement region may be irregular. \AMReX\space can only deal with logically cubical grids, thus a refinement region will be replaced by a new grid consisting of coarse and finer boxes on the region. Due to this ``discretization'' of the refinement region, more cells will be refined than necessary. Refining too many points becomes obviously inefficient. Conversely, refining too few points is also non-ideal, as it leads to a grid structure consisting of many small boxes. ``Grid efficiency'' then refers to the ratio of the points that need to be refined over the points that actually are refined in the new grid structure. This steers AMReX's eagerness to combine smaller boxes into larger ones by refining more points. The default value for this parameter is $0.7$

\subsubsection{\texttt{CarpetX::max\_tile\_size\_\{xyz\}}}

Controls tile sizes. In order to be cache efficient, tiling is disabled in the \texttt{x} direction because these elements are not stored contiguously in memory. The default values for \texttt{max\_tile\_size} are given in Tab.~\ref{tab:tile_sizes}

\begin{table}[h]
  \centering
  \begin{tabular}{cc}
    Parameter                   & Default value        \\\hline\hline
    \texttt{max\_tile\_size\_x} & $1024000$ (disabled) \\
    \texttt{max\_tile\_size\_y} & $16$                 \\
    \texttt{max\_tile\_size\_z} & $32$                 \\\hline\hline
  \end{tabular}
  \caption{\CarpetX's tile size control parameters and their default values.}
  \label{tab:tile_sizes}
\end{table}
