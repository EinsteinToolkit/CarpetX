% *======================================================================*
%  Cactus Thorn template for ThornGuide documentation
%  Author: Ian Kelley
%  Date: Sun Jun 02, 2002
%  $Header$
%
%  Thorn documentation in the latex file doc/documentation.tex
%  will be included in ThornGuides built with the Cactus make system.
%  The scripts employed by the make system automatically include
%  pages about variables, parameters and scheduling parsed from the
%  relevant thorn CCL files.
%
%  This template contains guidelines which help to assure that your
%  documentation will be correctly added to ThornGuides. More
%  information is available in the Cactus UsersGuide.
%
%  Guidelines:
%   - Do not change anything before the line
%       % START CACTUS THORNGUIDE",
%     except for filling in the title, author, date, etc. fields.
%        - Each of these fields should only be on ONE line.
%        - Author names should be separated with a \\ or a comma.
%   - You can define your own macros, but they must appear after
%     the START CACTUS THORNGUIDE line, and must not redefine standard
%     latex commands.
%   - To avoid name clashes with other thorns, 'labels', 'citations',
%     'references', and 'image' names should conform to the following
%     convention:
%       ARRANGEMENT_THORN_LABEL
%     For example, an image wave.eps in the arrangement CactusWave and
%     thorn WaveToyC should be renamed to CactusWave_WaveToyC_wave.eps
%   - Graphics should only be included using the graphicx package.
%     More specifically, with the "\includegraphics" command.  Do
%     not specify any graphic file extensions in your .tex file. This
%     will allow us to create a PDF version of the ThornGuide
%     via pdflatex.
%   - References should be included with the latex "\bibitem" command.
%   - Use \begin{abstract}...\end{abstract} instead of \abstract{...}
%   - Do not use \appendix, instead include any appendices you need as
%     standard sections.
%   - For the benefit of our Perl scripts, and for future extensions,
%     please use simple latex.
%
% *======================================================================*
%
% Example of including a graphic image:
%    \begin{figure}[ht]
% 	\begin{center}
%    	   \includegraphics[width=6cm]{MyArrangement_MyThorn_MyFigure}
% 	\end{center}
% 	\caption{Illustration of this and that}
% 	\label{MyArrangement_MyThorn_MyLabel}
%    \end{figure}
%
% Example of using a label:
%   \label{MyArrangement_MyThorn_MyLabel}
%
% Example of a citation:
%    \cite{MyArrangement_MyThorn_Author99}
%
% Example of including a reference
%   \bibitem{MyArrangement_MyThorn_Author99}
%   {J. Author, {\em The Title of the Book, Journal, or periodical}, 1 (1999),
%   1--16. {\tt http://www.nowhere.com/}}
%
% *======================================================================*

% If you are using CVS use this line to give version information
% $Header$

\documentclass{article}

% Use the Cactus ThornGuide style file
% (Automatically used from Cactus distribution, if you have a
%  thorn without the Cactus Flesh download this from the Cactus
%  homepage at www.cactuscode.org)
\usepackage{../../../../doc/latex/cactus}

\begin{document}

% The author of the documentation
\author{Steven R. Brandt \textless sbrandt@cct.lsu.edu\textgreater \\
        Roland Haas \textless rhaas@illinois.edu\textgreater}

% The title of the document (not necessarily the name of the Thorn)
\title{CarpetXRegrid}

% the date your document was last changed, if your document is in CVS,
% please use:
%    \date{$ $Date$ $}
% when using git instead record the commit ID:
%    \date{\gitrevision{<path-to-your-.git-directory>}}
\date{July 30 2024}

\maketitle

% Do not delete next line
% START CACTUS THORNGUIDE

% Add all definitions used in this documentation here
%   \def\mydef etc

% Add an abstract for this thorn's documentation
\begin{abstract}

Provide a grid function, \code{regrid\_error}, to the \code{CarpetX} driver.
This grid function is used by \code{CarpetX} and \code{AMReX} to determine
where to refine the grid.

\end{abstract}

% The following sections are suggestive only.
% Remove them or add your own.

\section{Introduction}

\code{AMReX}~\cite{CarpetX_CarpetXRegrid_AMReX_JOSS} and thus \code{CarpetX}
implement Berger-Oliger~\cite{CarpetX_CarpetXRegrid_Berger:1984zza} type
structured mesh refinement.  Grid \emph{cells} are tagged for refinement in
\code{CarpetX} using the overloaded virtual function \code{AmrCore::ErrorEst}. 

This thorn, in combination with \code{CarpetX}, provides a covenient way for
\code{Cactus} thorns to label cells needed refinement via a cell centered
grid function \code{regrid\_error} whose value controls if a cell is marked
for refinement or not.

A cell is marked for refinement if the value found in \code{regrid\_error}
exceeds the threshold in parameter \code{CarpetX::regrid\_error\_threshold}
and marked as not requiring refinement otherwise.

A client thorn wishing to control refinement should schedule a function in
schedule bins \code{POSTINITIAL} and \code{POSTSTEP}. See thorn
\code{ErrorEstimator} for an example.

\section{Numerical Implementation}

This thorn only provides access to the tagging grid function
\code{regrid\_error} and initializes it to zero at the beginning of the
simulation.

\section{Using This Thorn}

To use this thorn, make sure to \texttt{inherit} from \code{CarpetXRegrid} to
gain access to the public grid function \code{regrid\_error}. 
A client thorn wishing to control refinement should then  schedule a function in
schedule bins \code{POSTINITIAL} and \code{POSTSTEP} which should loop over
the \texttt{interior} of cell centered grid function \code{regrid\_error}
marking each cell for refinement or no refinement as needed. See thorn
\code{ErrorEstimator} for an example.

In the future, to allow multiple independent regridding criteria to contribute
to mesh refinement, grid function \code{regrid\_error} may be interpreted as a
boolean flag with true / false indicated by values of $1.0$ and $0.0$ (and all
other values considered invalid). In the future this will require that a thorn
should never reset non-zero value back to zero.

\subsection{Obtaining This Thorn}

This thorn is included in the \code{CarpetX} arrangement which is part of the
Einstein Toolkit. 

\subsection{Special Behaviour}

This thorn will initialize the error estimator grid function
\code{regrid\_error} to zero at the beginning of a simulation but does not
change or use its values otherwise.

In the future it may reset the error estimator grid function before client
thorns mark cells for refinement.

\subsection{Interaction With Other Thorns}

This thorn only provides access to \code{regrid\_error}, while the actual
regrid happens inside of \code{CarpetX} and \code{AMReX}. 

\subsection{Examples}

Please consult thorn \code{ErrorEstimator} for an example on how to use this
thorn.

\begin{verbatim}
extern "C" void MyRegrid_EstError(CCTK_ARGUMENTS) {
  DECLARE_CCTK_ARGUMENTSX_MyRegrid_EstError;
  DECLARE_CCTK_PARAMETERS;

  grid.loop_all_device<1, 1, 1>(
      grid.nghostzones,
      [=] CCTK_DEVICE(const Loop::PointDesc &p)
          CCTK_ATTRIBUTE_ALWAYS_INLINE {
              regrid_error(p.I) = fabs(p.x) < 42. ? 1.0 : 0.0;
      }
  );
}
\end{verbatim}

\subsection{Support and Feedback}

For questions and bug reports please use the Einstein Toolkit users mailing
list \url{users@einsteintoolkit.org} and bug tracker
\url{https://trac.einsteintoolkit.org}.

\begin{thebibliography}{9}
\bibitem{CarpetX_CarpetXRegrid_AMReX_JOSS} Zhang, W., Almgren, A., Beckner, V., Bell, J., Blaschke, J., Chan, C., Day, M., Friesen, B., Gott, K., Graves, D., Katz, M., Myers, A., Nguyen, T., Nonaka, A., Rosso, M., Williams, S. \& Zingale, M. AMReX: a framework for block-structured adaptive mesh refinement. {\em Journal Of Open Source Software}. \textbf{4}, 1370 (2019,5), https://doi.org/10.21105/joss.01370
\bibitem{CarpetX_CarpetXRegrid_Berger:1984zza} M.~J.~Berger and J.~Oliger, ``Adaptive Mesh Refinement for Hyperbolic Partial Differential Equations,'' J. Comput. Phys. \textbf{53} (1984), 484 doi:10.1016/0021-9991(84)90073-1
\end{thebibliography}

% Do not delete next line
% END CACTUS THORNGUIDE

\end{document}
